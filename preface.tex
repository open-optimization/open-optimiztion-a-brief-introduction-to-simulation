These course notes were originally written to support the course
Math 208W (Introduction to Operations Research) at Simon Fraser University
(British Columbia, Canada).
They are designed to for students with minimal mathematical background to get a hands-on
taste of Operations Research, a branch of mathematics which treats large-scale decision problems.

The focus of the course is on exploring such problems, which are too big for
analysis via pen-and-paper methods: indeed, simply collecting and managing
the data are a big part of the challenge.  The first part of the course
studies deterministic models, based on linear programming and extensions,
following Baker's \emph{Optimization Modeling with Spreadsheets}~\cite{baker2015}.  
This book suits the purpose quite well, as it assumes little in terms of prerequisites
and provides many worked examples and exercises, including data files on a supporting Website.  
The examples and exercises are written in \texttt{Microsoft Excel}, which is ubiquitous in the user
community, and a skill that Operations Researchers hoping to make an impact in practice should acquire.

For the second part of the course, we move on to non-deterministic models,
that incorporate an element of \emph{simulation}.  These notes are designed to
support this part of the course.
The notes continue in the style of Baker, with light 
prerequisites (Introductory Calculus) and extensive, \texttt{Excel}-based examples
and exercises, supported by data files.  The aim is not a rigorous presentation,
but rather an exposure to the basic ideas of simulation and an invitation to experiment.  

\vspace{4mm}
Chapter~\ref{sec1} begins the discussion with an introduction to random numbers
and how to generate them.  Chapter~\ref{sec2} then shows how a simple random
source can be used to generate different types of distributions that may be
useful in simulations.  Chapter~\ref{sec3} proceeds to use these distributions
to build some very basic simulations involving coin-flipping and random walks,
while Chapter~\ref{se:queueing} builds on this to examine simple queues and 
discrete-event simulations.  

\vspace{4mm}
There are several texts that introduce simulation.  We mention some which are
good sources for further reading, though they often require more extensive
mathematical prerequisites.  Hillier and Liebermann~\cite{hl2010} provide an
extensive introduction to Operations Research, including a chapter on
simulation.  Simulation textbooks include Bertsimas and Freund~\cite{bf2004},
Banks et al.~\cite{banks2010}, Guiasu~\cite{guiasu2009}, Law~\cite{law2015},
Leemis and Park~\cite{leemis2006} and Parlar~\cite{parlar2000}.

\vspace{4mm}
This work was started with support from a Teaching and Learning Development Grant (TLDG)
through the Institute for the Study of Teaching and Learning in the Disciplines
(ISTLD) at Simon Fraser University.  We thank the ISTLD staff who helped with
the project, including Cheryl Amundsen and Laura D'Amico for encouragement
and discussion, and the other participants in our workshop for feedback.

We thank Ismael Martinez and the students who took Math 208W \textit{Introduction to Operations Research}
at SFU for 2017-2019 for their comments and suggestions.

\vspace{4mm}
These notes are a work in progress, and comments, corrections and suggestions are encouraged.
The authors' contact details are available on their Webpages.\footnote{\url{https://www.utm.utoronto.ca/math-cs-stats/faculty-staff/yusun-dr-timothy} and \\ 
\url{http://people.math.sfu.ca/~tamon/}.}

\thispagestyle{plain} %% Or odd "Contents" header pops up.
